\section{Related Work}
\label{relatedwork}

Volume Visualization enables viewing of three-dimensional data by rendering
sampled function of three spatial dimensions and projecting the translucent
volume onto a 2D projection plane. While many techniques exist for volume
rendering such as marching cubes, ~\citep{lorensen1987marching},
image splatting  ~\citep{westover1990footprint}, texture slicing
[~\citep{rezk2000interactive}, ~\citep{engel2001high}], and ray casting
[~\citep{hsu1993segmented}, ~\citep{ma1995parallel}, ~\citep{ma1997scalable},
~\citep{heng2006gpu}]. ray casting has become one of the most used ones on
modern graphics hardware. Ray casting technique provides a higher quality
of rendering and many ways to optimize the rendering of the volume such
as early ray termination and space leaping ~\citep{yagel1993accelerating}.

While the Ray Casting a well-known technique and various implementations
exists, the challenge of providing high-quality volume visualization with
dataset of varying size, spacing, and format, with or without the geometry
rendering, on multiple platforms (Desktop, Cave, Head Mounted Displays (HMD))
and Operative Systems (Apple, Linux, Window) at interactive speed is still a challenge.

Few open-source implementations exist such as Voreen~\citep{MRMH09} which
initiated at Department of Computer Science at the University of Münster, Germany in 2004.
It provides features such as Isosurface Rendering,
Maximum Intensity Projection (MIP)~\citep{wallis1989three},
support for 1D and 2D transfer functions, and support for different illumination models
(Phong, Tone, Ambient Occlusion). Both Voreen and VTK uses data-flow networks. However,
Voreen is Volume Visualization library whereas VTK is a Scientific Visualization
library and provides better support for Geometry and Gridded datasets. While Voreen
provides a rapid prototyping environment, VTK volume visualization aims for production
quality, performance, and Ubiquitousness. Additionally, Voreen
only supports high-end desktop devices where as VTK supports multiple platforms
natively enabling researchers bridging the gap between academic research
and open source and industrial contributions.

Another open-source volume rendering ImageVis3D is being developed by the researchers
at the University of Utah~citep{SCI:ImageVis3D}. ImageVis3D is an application as opposed
to a library and provides support for large volume data visualization, volume visualization
on a desktop and mobile device, MIP, 1D and 2D transfer functions amongst many others.
It does not support Virtual Reality environments and the flexibility of a data-flow networks.
Few other data type specific volume rendering libraries such as Voxx~\citep{clendenon2002voxx}
and ClearVolume~\cite{royer2015clearvolume} provides visualization of biological and
light-sheet microscopy. PyMol ~\cite{delano2002pymol} provide volume visualization capabilities
only to molecular datasets. Finally, hardware architecture specific volume visualiation libraries
such as OSPray ~\cite{wald2017ospray} and NVIDIA Optix ~\citep{parker2010optix} provide fast
large volume data visualization capabilities on Intel CPU and Nvidia GPU's but lack support
for non-native hardware.

Other than Voreen and ImageVis3D, many volumes rendering APIs exist, however, except for
Voreen, advanced volume visualization research performed by the industry and
academic communities are not publicly available. By providing an open source volume rendering
engine that supports multiple platforms natively,  researchers can use the engine as a bridge
between academic research and open source and industrial contributions.

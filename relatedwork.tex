\section{Related Work}
\label{relatedwork}

Volume visualization offers a two-dimensional view of three-dimensional data by
sampling through the volume and projecting it on a 2D projection plane.  While
many techniques exist for volume rendering such as marching
cubes~\citep{lorensen_marching_1987}, image
splatting~\citep{westover_footprint_1990}, texture
slicing~\citep{rezk-salama_interactive_2000, engel_high-quality_2001} and ray
casting~\citep{hsu_segmented_1993, ma_parallel_1995, ma_scalable_1997,
heng_gpu-based_2005}, ray casting has become one of the most used ones on modern
graphics hardware. Ray casting technique provides a higher quality of rendering
and various options to optimize the rendering performance such as early ray
termination and space leaping~\citep{yagel_accelerating_1993}.

While ray casting is a well-known technique and various implementations exist,
providing high-quality volume visualization for datasets
varying in size, spacing and format, with or without geometry rendering, on
multiple platforms (Desktop, Cave, Head Mounted Displays (HMD)) and operating
systems (Apple, Linux, Window) at interactive speeds is still a challenge.

Few open source software systems exist for volume visualization, such as
Voreen~\citep{meyer-spradow_voreen:_2009} which initiated at Department of
Computer Science at the University of M\"unster, Germany in 2004.  It provides
features such as Isosurface Rendering, Maximum Intensity Projection
(MIP)~\citep{wallis_three-dimensional_1989}, support for 1D and 2D transfer
functions, and support for different illumination models (Phong, Tone, Ambient
Occlusion). Both Voreen and VTK use data-flow networks. However, Voreen is a
Volume Visualization library whereas VTK is a Scientific Visualization library
and provides better support for geometry and gridded datasets. While Voreen
provides a rapid prototyping environment, VTK volume visualization aims for
production quality, performance, and ubiquitousness. Additionally, Voreen only
supports high-end desktop devices where as VTK supports multiple platforms
natively, bridging the gap between academic research and open source and
industrial contributions.

Another example of an open source volume rendering software system is
ImageVis3D~\citep{cibc_imagevis3D:_2016}, that is being developed by the
researchers at the University of Utah. ImageVis3D is an application as opposed
to a library and provides support for large volume data, desktop as well as
mobile device, MIP, 1D and 2D transfer functions amongst many others.  It does
not support Virtual Reality environments and the flexibility of a data-flow
networks. Other data type specific volume rendering libraries such as
Voxx~\citep{clendenon_voxx:_2002} and ClearVolume~\citep{royer_clearvolume:_2015}
provide visualization of biological and light-sheet microscopy.
PyMOL~\citep{schrodinger_llc_pymol_2015} provides volume visualization
capabilities for molecular datasets only. Finally, hardware architecture specific
volume visualization libraries such as OSPRay~\citep{wald_ospray_2017} and
NVIDIA\textsuperscript{\textregistered}
Optix\textsuperscript{\texttrademark}~\citep{parker_optix:_2010} provide fast
large volume data visualization capabilities on Intel CPU and Nvidia GPU's but
lack support for non-native hardware.

Many volume rendering APIs exist apart from Voreen and ImageVis3D. However,
except for Voreen, advanced volume visualization research performed by the
industry and academic communities are not publicly available. By providing an
open source volume rendering engine that supports multiple platforms natively,
we hope to bridge the gap between academic research and industrial applications
in the field of volume visualization.

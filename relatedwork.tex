\section{Background}
\label{relatedwork}

Volume visualization provides a two-dimensional view of three-dimensional
uniformly sampled data by sampling and projecting the data onto a 2D
projection plane.  While many techniques exist for volume rendering, e.g.
marching cubes~\citep{lorensen_marching_1987}, image
splatting~\citep{westover_footprint_1990}, texture
slicing~\citep{rezk-salama_interactive_2000, engel_high-quality_2001}, and ray
casting~\citep{hsu_segmented_1993, ma_parallel_1995, ma_scalable_1997,
heng_gpu-based_2005}, ray casting has become pervasive on modern graphics
hardware. Ray casting techniques yield higher quality rendering, and provide
a variety of options to optimize the rendering performance, such as early ray
termination and space leaping~\citep{yagel_accelerating_1993}.

While ray casting is a well-known technique with many implementations, challenges still exist in
providing high-quality volume visualization for datasets of adequate
resolution, supporting a variety of spacing and formats, with or without
intermixed geometry rendering, on multiple platforms (desktop, Cave, Head
Mounted Displays (HMD)) and across operating systems (Apple, Linux, Window), at
interactive speeds.

A few open source software systems exist for volume visualization, such as
Voreen~\citep{meyer-spradow_voreen:_2009} which began at the Department of
Computer Science at the University of M\"unster, Germany, in 2004.  It provides
features such as isosurface rendering, maximum intensity projection
(MIP)~\citep{wallis_three-dimensional_1989}, support for 1D and 2D transfer
functions, and support for different illumination models (Phong, tone, ambient
occlusion). Both Voreen and VTK use data-flow networks. However, Voreen is a
volume visualization library whereas VTK is a scientific visualization library
which provides better support for geometry and gridded datasets. While Voreen
provides a rapid prototyping environment, VTK volume visualization aims for
production quality, performance, and ubiquitousness. Additionally, Voreen only
supports high-end desktop devices whereas VTK supports multiple platforms
natively, bridging the gap between academic research and practical application.
Other volume visualization capable tools such as ParaView, Slicer, and
MayaVi~\citep{ramachandran2011mayavi} all use VTK as the underlying
visualization library, and will benefit from our work.  ParaView has already
been switched to use this new approach, and Slicer is in the process of being
updated.

Another volume rendering-capable tool, Amira~\citep{stalling2005amira}, provides
multiple modules for volume visualization namely: voltex, volren, and volume
rendering module. In the context of our work, the volume rendering module of
Amira is most relevant. It provides support for GPU rendering, iso-surface, MIP,
and multi-volume support. In comparison to our work, it provides limited support
for data larger than main memory as it needs conversion to LDA format, lack of
other volume rendering modes such as MinIP, off-screen rendering, and
gradient-opacity and is closed-source.
OpenWalnut~\citep{eichelbaum2010openwalnut} is another open-source tools that
for visualization of medical and bio-signal data but provides very basic volume
visualization support, but does offer off-screen rendering support. Another
example of an open source volume rendering software system is
ImageVis3D~\citep{cibc_imagevis3D:_2016}, developed by the researchers at the
University of Utah. ImageVis3D is an application as opposed to a library and
provides support for large volume data, desktop as well as mobile devices, MIP,
1D and 2D transfer functions amongst many others.  It does not support Virtual
Reality environments and the flexibility of data-flow networks. Other data type
specific volume rendering libraries such as Voxx~\citep{clendenon_voxx:_2002}
and ClearVolume~\citep{royer_clearvolume:_2015} provide visualization of
biological and light-sheet microscopy.  PyMOL~\citep{schrodinger_llc_pymol_2015}
provides volume visualization capabilities for molecular datasets only. Finally,
hardware architecture specific volume visualization libraries such as
OSPRay~\citep{wald_ospray_2017} and NVIDIA\textsuperscript{\textregistered}
Optix\textsuperscript{\texttrademark}~\citep{parker_optix:_2010} provide fast
large volume data visualization capabilities on Intel CPU and Nvidia GPU's but
lack support for non-native hardware.

While it is difficult to compare our work with other available tools purely
based on the feature basis as features can be added or removed and also because
not all the features are documented by the tools, we will mark some fundamental
differences between our work and other available tools based on design choices
and architecture. Use of our VTK volume rendering requires that a user
constructs a VTK pipeline which comprises of  reader-filter-renderer. While the
pipeline approach provides the most flexibility in constructing workflows needed
for a particular study, it could also add some overhead in certain cases. For
example, some VTK imaging filters do not modify the input data which means that
when they perform any data operation, they copy the data and modify the newly
copied data. This means that the application will suffer a performance hit
specifically if the data is continuously generated or altered by the source of
the data. To mitigate this issue, it is essential that users understand the VTK
pipeline and its components well in advance.

Many volume rendering APIs exist apart from Voreen and ImageVis3D. However,
except for Voreen, advanced volume visualization research performed by the
industry and academic communities are not publicly available. By providing an
open source volume rendering engine that supports multiple platforms natively,
we hope to bridge the gap between academic research and industrial applications
in the field of volume visualization.

%% 
%% Copyright 2007, 2008, 2009 Elsevier Ltd
%% 
%% This file is part of the 'Elsarticle Bundle'.
%% ---------------------------------------------
%% 
%% It may be distributed under the conditions of the LaTeX Project Public
%% License, either version 1.2 of this license or (at your option) any
%% later version.  The latest version of this license is in
%%    http://www.latex-project.org/lppl.txt
%% and version 1.2 or later is part of all distributions of LaTeX
%% version 1999/12/01 or later.
%% 
%% The list of all files belonging to the 'Elsarticle Bundle' is
%% given in the file `manifest.txt'.
%% 

%% Template article for Elsevier's document class `elsarticle'
%% with numbered style bibliographic references
%% SP 2008/03/01

\documentclass[preprint,10pt,a4paper,5p,authoryear,twocolumn]{elsarticle}

%% Use the option review to obtain double line spacing
%% \documentclass[authoryear,preprint,review,12pt]{elsarticle}

%% For including figures, graphicx.sty has been loaded in
%% elsarticle.cls. If you prefer to use the old commands
%% please give \usepackage{epsfig}

%% The amssymb package provides various useful mathematical symbols
\usepackage{amssymb}
%% The amsthm package provides extended theorem environments
%% \usepackage{amsthm}

%% The lineno packages adds line numbers. Start line numbering with
%% \begin{linenumbers}, end it with \end{linenumbers}. Or switch it on
%% for the whole article with \linenumbers.
%\usepackage{lineno}

%% Set the bibliography style to be apalike i.e. author, year citation
\bibliographystyle{apalike}
%% By default, elsarticle loads the natbib package with round parameter showing
%% round brackets in citations. This can be changed to sqaure by setting the
%% citestyle
\setcitestyle{square}

%% This turns references into clickable hyperlinks.
\usepackage[bookmarks,
            colorlinks=true,
            linkcolor=black,
            citecolor=black,
            urlcolor=blue,
            linktocpage,
            pageanchor=true]{hyperref} %,colorlinks

%% The subcaption package is used for figure and table sub-captions
\usepackage[list=true]{subcaption}

%% For code 
\usepackage{listings}
\lstset{basicstyle=\scriptsize}

\journal{SoftwareX}

%% Setting the path to where the images can be found
\graphicspath{ {images/} }

\begin{document}

\begin{frontmatter}
%% Set the frontmatter to be two-column wide
  %\setlength\textwidth\columnwidth
%% Title, authors and addresses

%% use the tnoteref command within \title for footnotes;
%% use the tnotetext command for theassociated footnote;
%% use the fnref command within \author or \address for footnotes;
%% use the fntext command for theassociated footnote;
%% use the corref command within \author for corresponding author footnotes;
%% use the cortext command for theassociated footnote;
%% use the ead command for the email address,
%% and the form \ead[url] for the home page:
%% \title{Title\tnoteref{label1}}
%% \tnotetext[label1]{}
%% \author{Name\corref{cor1}\fnref{label2}}
%% \ead{email address}
%% \ead[url]{home page}
%% \fntext[label2]{}
%% \cortext[cor1]{}
%% \address{Address\fnref{label3}}
%% \fntext[label3]{}

  \title{Cross-platform ubiquitous volume rendering using programmable shaders in VTK for scientific and medical visualization}

\author{Aashish Chaudhary\corref{cor1}}
\ead{aashish.chaudhary@kitware.com}

\author{Sankhesh J. Jhaveri\corref{}}
\ead{sankhesh.jhaveri@kitware.com}

\author{Alvaro Sanchez}
\ead{alvaro.sanchez@kitware.com}

\author{Lisa S. Avila}
\ead{lisa.avila@kitware.com}

\author{Kenneth M. Martin}
\ead{ken.martin@kitware.com}

\author{David Lonie}
\ead{david.lonie@kitware.com}

\author{Marcus D. Hanwell}
\ead{marcus.hanwell@kitware.com}

\author{Will Schroeder}
\ead{will.schroeder@kitware.com}

\address{Kitware, Inc., 28 Corporate Drive, Clifton Park, NY 12065, USA}
\cortext[cor1]{Corresponding Author}


  \begin{abstract}
\label{abstract}
The Visualization Toolkit (VTK) is a popular cross-platform, open source
toolkit for scientific and medical data visualization, processing and
analysis. It supports a wide variety of data formats, algorithms and rendering
techniques for both polygonal and volumetric data. In particular VTK's volume
rendering module has long provided a comprehensive set of features such as
plane clipping, color and opacity transfer functions, lighting and other
controls needed for visualization. However, due to VTK's legacy OpenGL backend
and its reliance on a deprecated API, the system did not take advantage of the
latest improvements in graphics hardware and the flexibility of a programmable
pipeline. Additionally, this dependence on an antiquated pipeline posed
restrictions when running on emerging computing platforms thereby limited its
overall applicability. In response to these limitations, the VTK community
developed a new and improved volume rendering module, which not only produced
a modern GPU-based implementation, but augmented capabilities with new
features such as fast volume clipping, gradient magnitude based opacity
modulation, render to texture and hardware-based volume picking.
\end{abstract}


  \begin{keyword}
    Visualization Toolkit \sep
    Data analysis \sep
    Volume rendering \sep
    Scientific Visualization \sep
  \end{keyword}
%\title{Title/Name of your software}

%% use optional labels to link authors explicitly to addresses:
%% \author[label1,label2]{}
%% \address[label1]{}
%% \address[label2]{}

%\author{A. Author}

%\address{Your institute, some address}

%\begin{abstract}
%% Text of abstract 
%Ca. 100 words

%\end{abstract}

%\begin{keyword}
%% keywords here, in the form: keyword \sep keyword
%keyword 1 \sep keyword 2 \sep keyword 3

%% PACS codes here, in the form: \PACS code \sep code

%% MSC codes here, in the form: \MSC code \sep code
%% or \MSC[2008] code \sep code (2000 is the default)

%\end{keyword}

\end{frontmatter}

%\linenumbers

%% main text

%  Description of your software in maximum 6 pages.
%  
%  \section{Motivation and significance}
%  \label{}
%  
%  Introduce the scientific background and the motivation for developing the software.
%  
%  Explain why the software is important, and describe the exact (scientific) problem(s) it solves.
%  
%  Indicate in what way the software has contributed (or how it will contribute in the future) to the process of scientific discovery; if available, this is to be supported by citing a research paper using the software.
%  
%  Provide a description of the experimental setting (how does the user use the software?).
%  
%  Introduce related work in literature (cite or list algorithms used, other software etc.).
%  
%  
%  \section{Software description}
%  \label{}
%  
%  Describe the software in as much as is necessary to establish a vocabulary needed to explain its impact. 
%  
%  \subsection{Software Architecture}
%  \label{}
%  
%  Give a short overview of the overall software architecture; provide a pictorial component overview or similar (if possible). If necessary provide implementation details.
%  
%  \subsection{Software Functionalities}
%  \label{}
%  
%  Present the major functionalities of the software.
%  
%  \subsection{Sample code snippets analysis (optional)}
%  \label{}
%  
%  \section{Illustrative Examples}
%  \label{}
%  
%  Provide at least one illustrative example to demonstrate the major functions.
%  
%  Optional: you may include one explanatory video that will appear next to your article, in the right hand side panel. (Please upload any video as a single supplementary file with your article. Only one MP4 formatted, with 50MB maximum size, video is possible per article. Recommended video dimensions are 640 x 480 at a maximum of 30 frames/second. Prior to submission please test and validate your .mp4 file at $ http://elsevier-apps.sciverse.com/GadgetVideoPodcastPlayerWeb/verification$. This tool will display your video exactly in the same way as it will appear on ScienceDirect.).
%  
%  \section{Impact}
%  \label{}
%  
%  \textbf{This is the main section of the article and the reviewers weight the description here appropriately}
%  
%  Indicate in what way new research questions can be pursued as a result of the software (if any).
%  
%  Indicate in what way, and to what extent, the pursuit of existing research questions is improved (if so).
%  
%  Indicate in what way the software has changed the daily practice of its users (if so).
%  
%  Indicate how widespread the use of the software is within and outside the intended user group.
%  
%  Indicate in what way the software is used in commercial settings and/or how it led to the creation of spin-off companies (if so).
%  
%  \section{Conclusions}
%  \label{}
%  
%  Set out the conclusion of this original software publication.
%  
%  \section*{Acknowledgements}
%  \label{}
%  
%  Optionally thank people and institutes you need to acknowledge. 
%  
%  %% The Appendices part is started with the command \appendix;
%  %% appendix sections are then done as normal sections
%  %% \appendix
%  
%  %% \section{}
%  %% \label{}

\section{Introduction}
\label{introduction}
VTK is an open source cross-platform software system used for scientific data
processing, analysis, and visualization. It was originally developed to
supplement a textbook on object oriented computer graphics
programming~\citep{schroeder_visualization_2006, geveci_vtk_2012}.  Since its
inception, VTK has a long history of volume rendering and, unfortunately, that
history was evident in the large collection of classes used to render volumes.
While these methods were state-of-the-art at the time they were implemented,
given VTK's 20+ year history, many of these methods are now obsolete. Recently,
there has been a major effort~\citep{hanwell_visualization_2015} undertaken to
re-write VTK's rendering backend originally based on a legacy, now-deprecated
OpenGL API to one based on a modern programmable pipeline OpenGL
API~\citep{shreiner_opengl_2013}. This new rendering subsystem was designed to
support the latest technological advances in the graphics hardware industry.
Consequently one of the objectives of the work was to consolidate the number of
volume mappers to two: one supporting accelerated rendering using the Graphics
Processing Unit (GPU) and a second, parallel implementation on the Central
Processing Unit (CPU). In addition, a~\texttt{vtkSmartVolumeMapper} class was
added to assist application developers provide automatic run-time selection of
the appropriate volume rendering techniques based on system configuration
(see~\Autoref{fig:inheritancegraph}).

\begin{figure}[ht]
  \centering
  \includegraphics[width=\columnwidth]{vtk_volume_pipeline}%
  \caption{VTK pipeline for volume rendering which is similar to VTK polygonal
    rendering with differences such as transfer functions are defined on the
    property object.}
  \label{fig:pipeline}
\end{figure}%

Thus a primary objective of this OpenGL modernization effort, and the
subject of this article, was to create a cross-platform, multi-functional,
high-performance volume renderer supporting both serial and parallel execution
modes capable of supporting such complex applications as
ParaView~\citep{ahrens_paraview:_2005,ayachit_paraview_2015} or 3D
Slicer~\citep{fedorov_3d_2012}.
In addition our effort addressed the following key requirements:

\begin{itemize}
  \item Enable ubiquitous, cross-platform, high-performance support for volume
    rendering across all major operating systems and computing environments
    (desktop, VR, mobile).

  \item Ensure that the new volume visualization subsystem provides support for
    VTK-based pipeline and data-flow networks, thereby providing a flexible
    system addressing a wide variety of scientific and medical data
    visualization use-cases.

  \item Provide a variety of useful features at interactive frame rates such as
    clipping, cropping, gradient opacity, mixed geometry-volume translucent
    rendering, and on-demand shader composition.
\end{itemize}

\begin{figure}[htb!]
  \centering
  \includegraphics{inheritancegraph}%
  \caption{Graph depicting C++ class hierarchy for
    the~\texttt{vtkGPUVolumeRayCastMapper}}
  \label{fig:inheritancegraph}
\end{figure}%

To achieve this, we created a replacement for the OpenGL fixed pipeline
based~\texttt{vtkGPUVolumeRayCastMapper}. The new mapper, which shares the same
name but uses a modern OpenGL programmable pipeline, can be used
via~\texttt{vtkSmartVolumeMapper} or instantiated directly. Availability of the
new mapper with the new OpenGL-VTK implementation improves the management of
textures in the mapper and benefits both forms of rendering (geometry and
volume) by sharing common code between them. While volume ray-casting itself is
a well-known technique, developing a volume renderer that works with variety of
data formats and types, supports many essential features for medical and
scientific computing, works across all major computing platforms, and performs
well at interactive frame rates with very large datasets was a challenging task
that required an in-depth knowledge of the data, graphics pipeline, VTK
framework, and user requirements.  In the next section, we describe the
technical details behind the effort to produce the resulting modern,
cross-platform volume renderer delivered to the open source VTK community.

\subsection{Approach}
The new vtkGPUVolumeRayCastMapper uses a ray casting technique ~\ref{raycasting} for volume rendering which is a state-of-the-art for volume rendering on modern graphics platforms. Algorithmically, at a high level, it is similar to the older version of this class (although with a fairly different OpenGL implementation since that original class was first written over a decade ago and used GPU assembly code).  One of the main reason we chose to use ray casting due to the flexibility of this technique, which enables us to support all the features of the software ray cast mapper but with the acceleration of the GPU. Ray casting is an image-order rendering technique, with one or more rays cast through the volume per image pixel. VTK is inherently an object-order rendering system, where the GPU renders all graphical primitives (points, lines, triangles, etc.) represented by vtkProp(s) in the scene in one or more passes (with multiple passes needed to support advanced features such as depth peeling for transparency).

The image-order rendering process for vtkVolume is initiated when the front-facing polygons of the volume’s bounding box are rendered with a custom fragment program. This fragment program is used to cast a ray through the volume at each pixel, with the fragment location indicating the starting location for that ray. The volume and all the various rendering parameters are transferred to the GPU through the use of textures (3D for the volume, 1D for the various transfer functions) and uniform variables. Steps are taken along the ray until the ray exits the volume, and the resulting computed color and opacity are blended into the current pixel value. Note that volumes are rendered after all opaque geometry in the scene to allow the ray casting process to terminate at the depth value stored in the depth buffer for that pixel (and, hence, correctly intermix with opaque geometry).

In addition to providing supported features of the old mapper, the new mapper added new capabilities such as clipping on GPU,  gradient opacity, and volume picking amongst many others. In the next few sections, we will cover each of these features in detail.

\subsubsection{Single Pass}
\begin{figure}
\centering
\includegraphics[width=3in]{frontandback.png}
\caption{Front and back faces are rendered for start and end position of the ray.}
\label{fig:raycasting}
\end{figure}


\begin{figure}
\centering
\includegraphics[width=3in]{raycasting.jpg}
\caption{Implementing volume rendering using single-pass GPU ray casting.}
\label{fig:raycasting}
\end{figure}

In a ray-casting algorithm, the entry and the exit point into the volume is needed to determine when to stop the ray-marching. To determine the entry and the exit point, one approach is to render the geometry of the volume bounding box of the volume twice. In the first pass, the front face of the geometry is rendered and in the second pass the backface is rendered as shown in ~\ref{fig:frontandback}. Using the interpolated vertex position and texture lookup, the start and end positions is computed. Instead of this, in vtkGPUVolumeRayCastMapper, entry and exit points are computed based on the fact that the texture extents of the volume is within vec3(1.0), vec3(-1.0) range (and ~\ref{fig:raycasting}). The code below is showing the fragment shader piece that determines whether or not to stop marching the rays depending on the value of stop.
 
 \begin{lstlisting}[breaklines=true]
 bool stop = any(greaterThan(g_dataPos, 
                 ip_texMax)) ||
             any(lessThan(g_dataPos, 
                 ip_texMin));
 \end{lstlisting}
 
 The advantage of such approach is that it requires one less pass and is faster than other approaches since there is no texture generation or lookup happens for determining the termination of the ray.
 
 
\subsubsection{Dynamic Shader Generation}
In the new \texttt{vtkOpenGLGPUVolumeRayCastMapper} all the operations are performed on the GPU. The advantage of this approach was a more streamlined code that is easier to maintain and debug. This approach also provided an opportunity to rework how to support different features without having too many branches in the shader code or having to send all the options to the shader because that would have been detrimental to the performance. In the new \texttt{vtkOpenGLGPUVolumeRayCastMapper}, the shader is dynamically composed by the mapper. For this to work, we have introduced tags in a vertex, or fragment shader which are then replaced by the \texttt{vtkShaderComposer} depending on the option enabled or chosen by the application code. For instance, the skeleton fragment shader defines tags as shown below:
 
 \begin{lstlisting}
//VTK::Base::Dec

//VTK::Termination::Dec 
\end{lstlisting}

At the run time then \texttt{//VTK::Base::Dec} is then replaced by the code shown below. 

To define a structure, we have chosen a strategy that separates the tags in found category: 

1. Declaration (::Dec)
The tags belong to this group are meant to declare variables or function outside the main execution of the shader code. The variables defined are uniform, varying, and user-defined global variables. The functions defined are typically perform operations that are repetitive in nature such as computing color of a fragment. 

2. Initialization (::Init)
The tags belong to this group are meant to initialize variables inside the main execution function of the shader but before the ray-casting loop in the fragment shader. An example of such code includes computation of the initial position of ray and direction of ray traversal.

3. Implementation (::Impl)
The tags belong to this group are the variables or functions or the combination of both that perform the actual operation of clipping, cropping, shading, etc. on one, two, or four component volume data. The implementation code used local and global variables and optimized for performance reasons as they are executed as long as the ray is traversing inside the volume and didn't run into a termination condition which is checked every time.  

4. Exit (::Exit)
The tags belong to this group perform final computation such as the final color of the fragment. These tags are placed outside the ray-casting loop and typically contain numeric assignments.

\subsubsection{Lighting / Shading }
The old mapper supported only one light (due to limitations in OpenGL at the time the class was written). The \texttt{vtkFixedPointRayCastMapper} supports multiple lights, but only with an approximate lighting model, since gradients are precomputed and quantized, and shading is performed for each potential gradient direction regardless of fragment location. The new mapper accurately implemented the VTK lighting model to produce high quality images for publication. To support this, depending on the light type (point, directional, and positional), the lighting parameters are sent to the shader which then performs the per pixel lighting calculations. The number of lights is limited to six mostly because of the performance reasons as the interactive performance goes down significantly with each light added to the scene. The Phong shading lighting model is used for volume rendering lighting. Phong lighting requires normals which have to be computed for each fragment. The normal calculation is done by first computing the gradient and then scaling the gradient by the spacing between the cells. The gradient is computed by reading the scalar values form the neighboring cdells using the offset vector that stores the step size based on the bounds of the volume. 

\subsubsection{Volume Picking}
Picking in the context of volume rendering is the determination of which voxel user has clicked using a pointing device such as the mouse. Picking is one of most basic operations users performs to interact with a 3D scene. VTK legacy volume mappers support picking through an instance external to the mapper itself called ~\texttt{vtkVolumePicker}.  This class casts a ray into the volume and returns the point where the ray intersects an isosurface of a user specified opacity. This technique has certain limitations given that the picking class does not have enough information to correctly account for clipping, transfer functions and other parameters which define how the mapper renders internally, this reduces its reliability on the actual objects being picked. The inherent flexibility of the glsl-based implementation of the new mapper enables seamless integration with the ~\texttt{vtkHardwareSelector}'s interface, which allows a consistent selection of objects in a scene regardless of whether it is geometric or volumetric data.  Providing picking support directly within the fragment shader of the volume mapper ensures high selection accuracy even in situations where a volume intermixes with geometry in seemingly cumbersome ways or advanced features like when clipping planes are enabled (what you see is what you pick). Given the readily available picking styles supported by ~\texttt{vtkHardwareSelector} (e.g. ~\texttt{vtkAreaPicker}), it is possible to make a selection of a specific set of voxels.

\subsubsection{Volume Texture Streaming}
An intrinsic limitation of volume rendering is that the texture to be rendered does not always fit into the graphics memory of a system. This becomes increasingly important to address now that the new mapper provides support for mobile architectures.
A relatively simple method when dealing with a large volume is the divide-and-conquer approach, which is sometimes referred to as bricking. The volume is split into several blocks in such a way that a single sub-block (brick) fits completely into GPU memory.  Each sub-block is stored in main memory and streamed into GPU memory for a rendering pass one at a time (in a back-to-front manner for correct composition). The sub-blocks are rendered using the standard shader programs and alpha-blended with each other by OpenGL.
Streaming the volume as separate texture bricks certainly imposes a performance trade-off but acts as a graphics memory expansion scheme for devices that would not be able to render a higher quality volume otherwise.

\subsubsection{Dual Depth Peeling}
 
 
\subsubsection{Render to texture}
While the vtkGPUVolumeRayCastMapper uses a single pass rendering approach, it provides a render to texture mode to support advanced graphics applications. Render to texture is a graphics technique used in various multi-pass rendering pipelines for achieving advanced visual effects. It involves using the rendered output from a pass to modify / enhance the output of other rendering passes. The volume mapper allows rendering to an OpenGL FrameBufferObject (FBO) and obtaining the pixel data from the FBO via simple image retrieval calls. Two kinds of data can be obtained from  the mapper using the render to texture mode viz. color and depth information. The color data is simply the pixel representation of the rendered volume whereas depth data consists of a grayscale image depicting how deep each voxel is in the volume. When capturing the depth information, the depth of the first non-transparent voxel is captured.

\subsubsection{Mobile Support}
The move to OpenGL 3 and higher enabled volume rendering to support mobile devices (iOS and Android devices) as OpenGL ES 3.0 and higher supports 3D textures. Support for multiple touch events such as using two fingers to translate, rotate, and zoom the camera was added to support interactive rendering on mobile devices. New texture formats are added with logistics to enable / disable them based on the platform.. With minor feature-set exceptions, the new volume mapper works on mobile devices enabling developers to build sophisticated applications for the scientific community.

\subsubsection{Optimizations and Edge-Cases}
The new \texttt{vtkOpenGLGPUVolumeRayCastMapper} is more than just a ray cast mapper implementation. It is designed to work on multiple platforms and developed to perform volume rendering at interactive frame rates. To achieve interactive frame rates and to handle edge cases, we have implemented following optimizations in the new mapper. 

\begin{itemize}
\item Clipping plane optimization 
In VTK a user can place multiple planes at desired angles to clip the volume. This technique is essential for many medical use-cases. Since only one side of clip planes needs to be traversed, performing any sort of ray cast on the clipped side is wasteful. Hence a simple optimization is to move the starting point of the ray on the plane by projecting the ray onto the plane in the view direction. 

\item Eye Too Close to the Volume
When the eye is too close to the volume, the near plane could clip the volume visible bounds resulting in OpenGL clipping removing the geometry from rendering. In such cases the bounding box of the volume that defines the texture coordinates and hence define a surface to start casting the ray needs to be clipped. To handle such scenario, a plant-box interaction is added. So far evey move of the camera is funding invokved checking if the geomtry of the volume needs to be clipped.

\item Support for Double and Long Long Data Type


\end{itemize}
\subsection{Results}
\subsubsection{Cropping}
Cropping refers to 27 regions that defined by two planes along each coordinate axis of the volume and can be independently turned on (visible) or off (invisible) to produce a variety of different cropping effects, as shown in Figure~\ref{fig:cropping}. Cropping is implemented by determining the cropping region of each sample location along the ray and including only those samples that fall within a visible region.

\begin{figure}
\centering
\includegraphics[width=2.5in]{SphereCropping.png}
\caption{Figure 3. A sphere is cropped using two different configurations of cropping regions.}
\label{fig:cropping}
\end{figure}

\subsubsection{Wide Support of Data Types} 
The vtkGPURayCastMapper supports most data types such as short, int, float, and double and both point and cell data types. Bias are scale are computed and applied to the scalars in the fragment shader to normalize the scalars between 0-1 range. 

\subsubsection{Clipping}
A set of infinite clipping planes can be defined to clip the volume to reveal inner detail, as shown in Figure 4.  Clipping is implemented by determining the visibility of each sample along the ray according to whether that location is excluded by the clipping planes.A set of infinite clipping planes can be defined to clip the volume to reveal inner detail, as shown in Figure 4.  Clipping is implemented by determining the visibility of each sample along the ray according to whether that location is excluded by the clipping planes.

\begin{figure}
\centering
\includegraphics[width=2.5in]{HeadClippingOblique.png}
\caption{Figure 4. Top: An example of an oblique clipping plane. Bottom: A pair of parallel clipping planes clip the volume, rendered without (left) and with (right) shading.}
\label{fig:clipping}
\end{figure}

\subsubsection{Blending Modes}
The mapper supports composite blending, minimum intensity projection, maximum intensity projection, and additive blending. Each of these blending modes are useful for a particular use-case in medical computing. The most common one which is also the default is the composite blending mode. See Figure~\ref{fig:blendingmodes} for an example of composite blending, maximum intensity projection, and additive blending on the same data.

\begin{figure*}
\centering
\begin{subfigure}{.6\columnwidth}
    \includegraphics[width=\columnwidth]{TorsoBlendingComposite.png}
\end{subfigure}
\begin{subfigure}{.6\columnwidth}   
    \includegraphics[width=\columnwidth]{TorsoBlendingAdditive.png}
\end{subfigure} 
\begin{subfigure}{.6\columnwidth}
    \includegraphics[width=\columnwidth]{TorsoBlendingMIP.png}
\end{subfigure}
\caption{Rendering with and without gradient opacity transfer function.}
\label{fig:blendingmodes}
\end{figure*}

\subsubsection{Masking}
Both binary and label masks are supported. With binary masks, the value in the masking volume indicates visibility of the voxel in the data volume. When a label map is in use, the value in the label map is used to select different rendering parameters for that sample.  See Figure 5 for an example of label data masks.

\subsubsection{Opacity Modulated by Gradient Magnitude}
A transfer function mapping the magnitude of the gradient to an opacity modulation value can be used to essentially perform edge detection (de-emphasize homogenous regions) during rendering. See ~\ref{fig:gradient} for an example of rendering with and without the use of a gradient opacity transfer function.

\begin{figure*}
\centering
   \begin{subfigure}[b]{0.5\textwidth}
   \includegraphics[width=1\linewidth]{TorsoGradient.png}
   \caption{}
   \label{fig:Ng1} 
\end{subfigure}

\begin{subfigure}[b]{0.5\textwidth}
   \includegraphics[width=1\linewidth]{TorsoNoGradient.png}
   \caption{}
   \label{fig:Ng2}
\end{subfigure}

\caption{Rendering with and without gradient opacity transfer function.}
\label{fig:gradient}
\end{figure*}

\subsubsection{Mobile Support}
The move to OpenGL 3 and higher enabled volume rendering to support mobile devices (iOS and Android devices) as OpenGL ES 3.0 and higher supports 3D textures. Support for multiple touch events such as using two fingers to translate, rotate, and zoom the camera was added to support interactive rendering on mobile devices. New texture formats are added with logistics to enable / disable them based on the platform.. With minor feature-set exceptions, the new volume mapper works on mobile devices enabling developers to build sophisticated applications for the scientific community.

\subsubsection{Volume Picking}
Picking is defined in this context as the effort of determining which on-screen object a user has clicked on,  this is one of the most basic operations to interact with a 3D scene. VTK legacy volume mappers support picking through an instance external to the mapper itself called vtkVolumePicker.  This class casts a ray into the volume and returns the point where the ray intersects an isosurface of a user specified opacity. This technique has certain limitations given that the picking class does not have enough information to correctly account for clipping, transfer functions and other parameters which define how the mapper renders internally, this reduces its reliability on the actual objects being picked.
The inherent flexibility of the glsl-based implementation of the new vtkGPURayCastMapper enables seamless integration with the vtkHardwareSelector's interface, which allows a consistent selection of objects in a scene regardless of whether it is geometric or volumetric data.  Providing picking support directly within the fragment shader of the volume mapper ensures high selection accuracy even in situations where a volume intermixes with geometry in seemingly cumbersome ways or advanced features like when clipping planes are enabled (what you see is what you pick).
Furthermore, this initial implementation supports a higher picking granularity than only the volume object itself (vtkProp).  Given the readily available picking styles supported by vtkHardwareSelector (e.g. vtkAreaPicker), it is possible to make a selection of a specific set of voxels.

\subsubsection{Lighting / Shading }
The old GPU ray cast mapper supported only one light (due to limitations in OpenGL at the time the class was written). The vtkFixedPointRayCastMapper supports multiple lights, but only with an approximate lighting model, since gradients are precomputed and quantized, and shading is performed for each potential gradient direction regardless of fragment location. The new vtkGPURayCastMapper accurately implemented the VTK lighting model to produce high quality images for publication. Up To six lights are supported (point, directional, and positional). The number of lights are limited to six mostly because of the performance reasons as the interactive performance goes down significantly with each light added to the scene. 

\subsubsection{Volume Texture Streaming}
An intrinsic limitation of volume rendering is that the texture to be rendered does not always fit into the graphics memory of a system. This becomes increasingly important to address now that the new mapper provides support for mobile architectures.
A relatively simple method when dealing with a large volume is the divide-and-conquer approach, which is sometimes referred to as bricking. The volume is split into several blocks in such a way that a single sub-block (brick) fits completely into GPU memory.  Each sub-block is stored in main memory and streamed into GPU memory for a rendering pass one at a time (in a back-to-front manner for correct composition). The sub-blocks are rendered using the standard shader programs and alpha-blended with each other by OpenGL.
Streaming the volume as separate texture bricks certainly imposes a performance trade-off but acts as a graphics memory expansion scheme for devices that would not be able to render a higher quality volume otherwise.
\section{Future Work}
\label{future-work}
As a result of this work, we have developed a replacement class for
vtkGPURayCastMapper that is more widely supported, faster, more easily
extensible, and supports majority of the features of the old class. In the
near future, our goal is to ensure that this mapper works as promised by
integrating it into existing VTK applications such as ParaView and 3D Slicer.
Once this integration and proofing process is complete, we plan on adding
additional, advanced features as described in the following.

\subsection{2D Transfer Functions}
\label{2d-transfer-functions}
Currently, volume rendering in VTK uses three independent 1D transfer functions
to map scalar value to color, scalar value to opacity and gradient magnitude to
opacity. Increasing the number of parameters in a transfer function can improve
discrimination between structures in the volume data given that the combined
parametric information allows to disambiguate areas that fall within a given range
of those parameters simultaneously. Enhanced structure discrimination is beneficial
in medical image visualization where distinct tissues are approximately constant
in value and values transition smoothly from one tissue to the next. There is ongoing
work to support 2D transfer functions combining scalar value and gradient magnitude
(the scalar field variable and its first derivative) under the the DOE Office of
Science contract DE-SC0011385 grant.

\subsection{Overlapping Volumes}
\label{overlapping-volumes}
It is currently possible to render overlapping volumes by taking advantage of
the up to four independent components supported by the mapper (each component
representing a different volume).
The limitation of this approach is that each of the overlapping volumes are
required to be sampled in the same grid, hence all of the volumes are required
to share the same dimensions.  Nonetheless, in order to extend the mapper to
support overlapping volumes sampled in grids with different dimensions, rays
can be cast through proxy geometry bounding the N overlapping volumes to be
rendered and separately sampling and compositing their fetched texture values in
the fragment shader.

\subsection{Improved Rendering of Labeled Data}
\label{improved-rendering-of-labeled-data}
Currently, VTK supports binary masks and only a couple of specific representations
of label mapping. We know that our community needs more extensive label mapping
functionality--especially for medical datasets. Labeled data requires careful
attention to the interpolation method used for various parameters. (Users may wish
to use linear interpolation for the scalar value to look up opacity, but select 
the nearest label to look up the color.) We
plan to solicit feedback from the VTK community to understand the sources of
labeled data and the application requirements for visualization of such data. We
then hope to implement more comprehensive labeled data volume rendering for both
the CPU and GPU mappers.

\section{Acknowledgements}
We would like to recognize the National Institutes of Health for sponsoring this work under the grant NIH R01EB014955 “Accelerating Community-Driven Medical Innovation with VTK.” We would like to thank Marcus Hanwell and Ken Martin, who are tirelessly modernizing VTK by bringing it to OpenGL 2.1 and mobile devices and who have been providing feedback on this volume rendering effort.
The Head and Torso datasets used in this article are available on the Web at http://www.osirix-viewer.com/datasets.

%% References:
\section{References}
\label{}
\bibliography{VTKVR}

\end{document}
\endinput
%%
%% End of file `SoftwareX_article_template.tex'.

\section{Application Areas}
\label{applicationareas}

One of the goals of our work is to support volume visualization on multiple
operating systems (Linux, Mac, Windows), devices (Workstation,
Virtual Reality, Mobile, Cluster), and for multiple domains (Scientific, Medical).
This is important as VTK is used by a large user base in different setups.
In the next few sub-sections, we have presented improvements to
our mapper for different domains and environments.

\subsection{Domains}
\label{domains}

\subsection{Devices and Environments}
Our mapper support rendering of volumes on all commonly used devices. This poses
a challenge as the mapper has to handle low computing power of mobile devices or
composition in a cluster environment and still perform at Interactive frame rates.
To support mobile environments, we have implemented following features:
\begin{enumerate}
  \item We added support in the RenderWindowInteractor for multitouch
  events. We currently have built-in support for receiving and handling
  multitouch events on iOS, Android, and Windows.
  \item Added check for OpenGL ES capabilities and accordingly managed
  formats that are not supported on mobile devices
  \item We added support for compiling VTK on iOS and Android. While still
  in a beta stage we have added in support for compiling VTK for Android and
  iOS platforms
\end{enumerate}

Various artifacts arise when doing parallel volume rendering (bricking). Artifacts
due to sampling, gradient computation, etc.  are seen at the edges of each of the bricks.
To address it, our mapper ensures the entry texture coordinate and the limit texture
coordinates are correctly adjusted to, and the ray step is scaled accordingly.

Finally, to perform consistently across various devices, our mapper uses two
critical information from VTK. First, the last frame-render time, and second,
the desired frame rate for the application. Using these two factors,  the mapper
then increases the sampling distance anytime users interacts with
the volume and identifies the desired framerate is not achievable at user-specified
sampling distance.


\label{devices}
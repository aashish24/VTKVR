\section{Future Work}
\label{future-work}
At this point, we have a replacement class for vtkGPURayCastMapper that is more
widely supported, faster, more easily extensible, and supports majority of the
features of the old class. In the near future, our goal is to ensure that this
mapper works as promised by integrating it into existing VTK applications such
as ParaView and Slicer. Once these tasks are complete, we have some ideas on new
features we would like to add to this mapper (outlined below). We would also
like to solicit feedback from folks using the VTK volume mappers. What features
do you need? Drop us a line at kitware@kitware.com, and let us know.

\subsection{2D Transfer Functions}
\label{2d-transfer-functions}
Currently, volume rendering in VTK uses two 1D transfer functions, mapping
scalar value to opacity and gradient magnitude to opacity. For some application
areas, better rendering results can be obtained by using a 2D table that maps
these two parameters into an opacity value. Part of the challenge in adding a
new feature such as this to volume rendering in VTK is simply the number of
volume mappers that have to be updated to handle it (either correctly rendering
according to these new parameters or at least gracefully implementing an
approximation). Once we
have reduced the number of volume mappers in VTK, then adding new features such
as this will become more manageable.

\subsection{Overlapping Volumes}
\label{overlapping-volumes}
It is currently possible to render overlapping volumes by taking advantage of
the up to 4 independent components supported by the mapper (each component
representing a different volume).
The limitation with this approach is that each of the overlapping volumes are
required to be sampled in the same grid, hence all of the volumes are required
to share the same dimensions.  Nonetheless, in order to extend the mapper to
support overlapping volumes sampled in grids with different dimensions, rays
should be casted through proxy geometry bounding the N overlapping volumes to be
rendered and separately sampling and compositing their fetched texture values in
the fragment shader.

\subsection{Improved Rendering of Labeled Data}
\label{improved-rendering-of-labeled-data}
Currently, VTK supports binary masks and only a couple of very specific versions
of label mapping. We know that our community needs more extensive label mapping
functionality – especially for medical datasets. Labeled data requires careful
attention to the interpolation method used for various parameters. (You may wish
to use linear interpolation for the scalar value to look up opacity, but,
perhaps, select the nearest label to look up the color.) We
plan to solicit feedback from the VTK community to understand the sources of
labeled data and the application requirements for visualization of this data. We
then hope to implement more comprehensive labeled data volume rendering for both
the CPU and GPU mappers.

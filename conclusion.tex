\section{Future Work}
\label{future-work}

As a result of this work, we have developed a replacement class for
\texttt{vtkGPUVolumeRayCastMapper} that is more widely supported, faster, more
easily extensible, and supports majority of the features of the old class. In
the near future, our goal is to ensure that the mapper works as promised with
existing VTK applications. Eventually, we plan on adding additional, advanced
features as described in the following sections.

\subsection{2D Transfer Functions}
\label{2d-transfer-functions}
Currently, volume rendering in VTK uses three independent 1D transfer functions
to map scalar value to color, scalar value to opacity and gradient magnitude to
opacity. Increasing the number of parameters in a transfer function can improve
discrimination between structures in the volume data given that the combined
parametric information allows to disambiguate areas that fall within a given
range of those parameters simultaneously. Enhanced structure discrimination is
beneficial in medical image visualization where distinct tissues are
approximately constant in value and values transition smoothly from one tissue
to the next. There is ongoing work to support 2D transfer functions combining
scalar value and gradient magnitude (the scalar field variable and its first
derivative) under the US Department of Energy (DOE) Office of Science (SC)
contract DE-SC0011385 grant.

\subsection{Overlapping Volumes}
\label{overlapping-volumes}
It is currently possible to render overlapping volumes by taking advantage of
the up to four independent components supported by the mapper (each component
representing a different volume).
The limitation of this approach is that each of the overlapping volumes are
required to be sampled in the same grid, hence all of the volumes are required
to share the same dimensions.  Nonetheless, in order to extend the mapper to
support overlapping volumes sampled in grids with different dimensions, rays
can be cast through proxy geometry bounding the N overlapping volumes to be
rendered and separately sampling and compositing their fetched texture values in
the fragment shader.

\subsection{Improved Rendering of Labeled Data}
\label{improved-rendering-of-labeled-data}
Currently, VTK supports binary masks and only a couple of specific
representations of label mapping. We know that our community needs more
extensive label mapping functionality--especially for medical datasets. Labeled
data requires careful attention to the interpolation method used for various
parameters. (Users may wish to use linear interpolation for the scalar value to
look up opacity, but select the nearest label to look up the color.) We plan to
solicit feedback from the VTK community to understand the sources of labeled
data and the application requirements for visualization of such data. We then
hope to implement more comprehensive labeled data volume rendering for both the
CPU and GPU mappers.
